
% @Author: winterzz1 1002658987@qq.com
% @Date: 2023-10-17 22:44:04
% @LastEditors: winterzz1 1002658987@qq.com
% @LastEditTime: 2023-10-17 22:52:23
% @FilePath: /chino-acm-template/latex/main.tex
% @Description: 模板latex-pdf

\documentclass{article}
\usepackage{color}
\usepackage{xcolor}
\usepackage{listings}
\usepackage{CJK}
\usepackage{blindtext}
\usepackage{caption}
\usepackage{lipsum}
\usepackage{geometry}
\usepackage{subfiles}
\usepackage{titlesec}
\usepackage{graphicx}
\usepackage{markdown}
\usepackage{fancyhdr}
\usepackage{float}

 \geometry{
 a4paper,
 total={170mm,257mm},
 left=20mm,
 top=20mm,
 }
\lstset{
  tabsize=4, 
  frame=lines, 
  commentstyle=\color{red!50!green!50!blue!50},
  rulesepcolor=\color{red!20!green!20!blue!20},
  keywordstyle=\color{blue!90}\bfseries, 
  showstringspaces=false,
  stringstyle=\tt, 
  keepspaces=true, 
  breakindent=50pt, 
  numbers=left,
  stepnumber=1,
  numberstyle=\tt, 
  basicstyle=\tt, 
  showspaces=false, 
  flexiblecolumns=true, 
  breaklines=true,
  breakautoindent=true,
  breakindent=4em, 
  escapeinside={`}{`},
  escapebegin=\begin{CJK*}{UTF8}{gbsn},escapeend=\end{CJK*},
  aboveskip=1em, 
  fontadjust,
  captionpos=t,
  texcl=true,
  extendedchars=false,columns=flexible,mathescape=true
}
\setlength{\parindent}{0pt}
\markdownSetup{fencedCode = true} %markdown内联代码框
\markdownSetup{hybrid = true} %markdown内联latex
\markdownSetup{ %markdown内联图片设置
  renderers = {
    image = {\begin{figure}[H]
      \centering

      \includegraphics[width=0.5\textwidth]{#3}%
      \ifx\empty#4\empty\else
        \caption{#4}\label{fig:#1}%
      \fi
    \end{figure}},
  }
}

\lstset{language=c++}%代码语言使用的是c++
\lstset{breaklines}%自动将长的代码行换行排版
\lstset{extendedchars=false}%解决代码跨页时,章节标题,页眉等汉字不显示的问题





\begin{document}


\begin{CJK*}{UTF8}{gbsn}




\date{2023-10-01}
\title{四糸智乃的算法模板}
\author{四糸智乃}

\maketitle
\begin{center}
\includegraphics[width=1.0\textwidth]{./resource/title.jpg}
\end{center}

\newpage
\newpage

\twocolumn

\tableofcontents

\newcommand{\currentsection}{}
\let\oldsection\section
\renewcommand{\section}[1]{\oldsection{#1}\renewcommand{\currentsection}{#1}}

\newcommand{\currentsubsection}{}
\let\oldsubsection\subsection
\renewcommand{\subsection}[1]{\oldsubsection{#1}\renewcommand{\currentsubsection}{#1}}

\newcommand{\currentsubsubsection}{}
\let\oldsubsubsection\subsubsection
\renewcommand{\subsubsection}[1]{\oldsubsubsection{#1}\renewcommand{\currentsubsubsection}{#1}}


\onecolumn
\newpage

\pagestyle{fancy}%清除原页眉页脚样式
\fancyhead{} % 初始化页眉
\fancyhead[C]{\textsl{\rightmark-\currentsubsection-\thepage}}
\renewcommand{\headrulewidth}{4pt}
\fancyfoot{} % 初始化页脚
\fancyfoot[L]{https://github.com/yoshichino/chino-acm-template}

\chapter{正文}

\section{数据结构}
\subfile{dataStructure/dataStructure}
\section{数学}
\subfile{math/math}

\section{教程}
\subfile{tutorial/tutorial}

\end{CJK*}
\end{document}

